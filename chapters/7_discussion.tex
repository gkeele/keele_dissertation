\chapter{Concluding remarks}\label{chap:discussion}

In this dissertation, statistical methods are developed and utilized that support the complete arc of an experiment involving genetic association within a multiparental population (MPP):
\begin{enumerate}
	\item Experimental design
	\item Association analysis and related fine-mapping and follow-up analyses
\end{enumerate}

MPP are experimental populations composed of individuals descended from more than two inbred founders. They have been developed in a number of animal systems, including but certainly not limited to mice \citep{Churchill2004,Churchill2012}, rats \citep{Hansen1984}, fly \citep{King2012a, Long2014, King2017, Najarro2017, Stanley2017}, and roundworm \citep{Noble2017}, as well as numerous plant systems \citep{Kover2009, Bandillo2013,Buckler2009,Bouchet2017,Mangandi2017,Tisne2017}. These populations are valuable resources for genetic experiments due to the greater extent of genetic and likely phenotypic variation they possess in comparison to traditional inbred strains and bi-parental crosses. Though analytical procedures developed in bi-parental populations have been successfully extended to MPP, such as Haley-Knott (HK) regression \citep{Haley1992,Martinez1992}, there is potential for modeling approaches that better accommodate these resources and can thus more efficiently and powerfully extract the underly biological inferences. The procedures and analyses presented were performed with data from MPP rodent populations (mice and rats), however, the fundamental concepts and ideas generalize to other organisms and their MPP. Herein is a summary of the findings and conclusions from the chapters of this dissertation.

\section{Experimental design}

\subsection{Using the diallel to select optimal bi-parental crosses to map QTL}

\textbf{Chapter \ref{chap:didact}} dealt with, in general terms, a Bayesian decision theoretic approach for the evaluation of a utility function based on pilot data across potential downstream experiments, with the intent to improve selection of experimental designs. In application, a Bayesian hierarchical model used the diallel, a unique MPP, as pilot data to characterize strain-level effects \citep{Lenarcic2012}, which were then the inputs into the utility functions for possible bi-parental crosses (F2 intercrosses, backcrosses (BC), and parent-of-origin effect reciprocal BC), implemented as the R package DIDACT (Diallel Informed Decision theoretic Approach for Crosses Tool). The utility function used in DIDACT was the power to detect a putative QTL underlying the strain-level effects, though simpler functions could be implemented. 

In practice, DIDACT was found to perform well in both Mendelian and complex phenotypes. For a largely Mendelian trait, body weight loss percentage after Influenza A infection, which is known to be largely driven by the gene \textit{Mx1} \citep{Maurizio2017}, DIDACT correctly favors bi-parental crosses that match a strain with null \textit{Mx1} allele with a strain with a functional allele, thus resulting in mapping populations with segregating alleles at the gene. For complex phenotypes, in which the strain-level effects are likely highly polygenic in nature, though the assumptions underlying the utility function are false and the nominal power biased upward, DIDACT favors crosses that match strains that are disparate in phenotype. Thus DIDACT provides a quantitative and principled approach to selecting bi-parental crosses that in practice will not deviate from good and standard practice of matching phenotypically distinct strains, while also allowing interesting strain-level effect combinations to inform the utility function.

\subsection{Simulated power to map QTL in the realized Collaborative Cross}

\cite{Valdar2006c} estimated power to map QTL in the Collaborative Cross (CC) through simulation, in which the recombinant inbred (RI) strain genomes and phenotype were simulated, based on the stated expectation of 1000 RI strains. Due to allelic incompatibilities many CC lines went extinct \citep{Shorter2017}, resulting in approximately 75 final strains. \textbf{Chapter \ref{chap:sparcc}} describes the R package SPARCC (Simulated Power Analysis in the Realized Collaborative Cross), which is designed to provide power calculations that are highly tailored to specific experiments in the actual finalized CC genomes, and thus assist researchers in designing their CC experiments. Additionally SPARCC can be used to explore the effect on power of various aspects of the experimental design: the number of CC strains and the number of replicate observations, and the underlying biology: QTL effect size, background strain effect size, QTL position, and allelic series \citep{Yalcin2005}, representing the number of function alleles and their distribution amongst the founder strains.

Based on large-scale simulations, SPARCC finds that increasing the number of CC strains is more important than increasing replicate observations, though both will improve mapping power. With respect to the allelic series, as the number of functional alleles increases, the power increases. An increase in background strain variance reduces mapping power, which replicate observations cannot improve. For QTL with fewer functional alleles than the number of founders (less than eight), the balance in how the alleles are distributed amongst the founders strongly affects power, with greater balance resulting in increased power. Summary power curves from SPARCC are also presented for general reference for researchers designing CC mapping experiments.

\section{Genetic association and related analyses}

\subsection{Accounting for haplotype uncertainty in QTL mapping of multiparental populations using multiple imputation}

In \textbf{Chapter \ref{chap:mi}} a haplotype-based QTL mapping approach is proposed that takes a multiple imputation (MI) approach to conservatively and stably test for QTL associations in comparison to the unstable associations observed using the standard approach for MPP, HK regression, also referred to as regression-on-probabilities (ROP) \citep{Haley1992,Martinez1992}. Simulations show that MI is conservative in comparison to ROP when the founder haplotype contributions are roughly balanced at loci across the genome. Problems arise for ROP when imbalanced founder haplotype contributions and haplotype uncertainty combine to produce strong correlations between the phenotype and near-zero probabilities for a founder haplotype that has been lost through genetic drift, as is observed frequently in a heterogenous stock (HS) rat data set as well as at some loci in the CC. Though more computationally intensive than ROP, MI is still feasible, as well as providing the ability to observe the fragility of association over imputations.

Also discussed is the problematic situation when founder haplotype contributions are imbalanced, but now with great certainty, resulting in haplotype parameters that are fit based on only a few individuals, representing highly leveraged data points. MI will not reduce strong associations that result from leveraged data points that have extreme outcomes. This situation actually represents the biased associations that result from fitting a fixed effect parameter to too few data points, and as such, shrinkage procedures should be used, either through variance components \citep{Wei2016} or through pseudo-observations. Though computationally less efficient than ROP, MI and shrinkage methods provide QTL mapping approaches in MPP when ROP fails.

\subsection{QTL mapping in outbred rat population with imbalanced founder allele frequencies}

\textbf{Chapter \ref{chap:hs_rats}} describes a QTL mapping analysis in the the HS population first described in \textbf{Chapter \ref{chap:mi}} that detected QTL for adiposity traits, specifically two QTL for retroperitoneal fat pads (RetroFat) and one QTL for body weight, and the subsequent fine-mapping analyses to identify candidate genes and variants. As this dissertation is largely focused on statistical methods for MPP, the summary of results will focus on the development and use of methods, rather than the specific genes and variants that were indentified. 

An imputed SNP association approach to QTL mapping was used rather than haplotype-based association because the HS had highly imbalanced founder haplotype contributions at most loci, as well as high levels of uncertainty in terms of distinguishing certain founder haplotypes. In such a context, imputed SNP association was found to be stable, and potentially more powerful for this population than the MI approach previously described.

Various models were used to fine-map the QTL regions. The LLARRMA-dawg method \citep{Sabourin2015} reduced the RetroFat chromosome 6 QTL interval from 6.14 Mb to 1.46 Mb. Founder haplotype effects were estimated with the Diploffect model \citep{Zhang2014}, which were used to prioritize variants in the region that matched the effect pattern. In particular, a strong effect from the WKY founder was detected for the RetroFat chromosome 6 QTL, which identified a cluster of genes with unique WKY alleles in the region. Protein modeling \citep{Prokop2017} was used to predict the effect of candidate variant alleles with respect to protein function, which supported the gene \textit{Adcy3} for the RetroFat chromosome 6 QTL, and is also supported in the literature \citep{Speliotes2010, Nordman2008, Stergiakouli2014, Wen2012}, and \textit{Prlhr} for the RetroFat chromosome 1 QTL. \textit{Grid2} was the primary candidate for the body weight chromosome 4 QTL based on previous literature \citep{Dietrich2013,Locke2015} and there only being two genes present in the region.

Finally, an integrative mediation analysis was used to fine-map the QTL regions as well, testing for the potential that the phenotypes are modulated partly through gene expression. The RetroFat chromosome 6 locus contained many co-localizing expression QTL (eQTL), many also driven by variants present in the WKY founder. Expression of the gene \textit{Krtcap3} was found to be a full mediator of the QTL effect on RetroFat. There was also evidence for the expression of the gene \textit{Slc30a} as partial mediator or suppressor. Essentially nothing is known about \textit{Krtcap3} based on the literature and bioinformatic resources. It is possible that the WKY effect in the region is multifactorial, the result of changes to the expression levels of multiple genes as well as protein function of \textit{Adcy3}.

\subsection{Detecting chromatin accessibility as a mediator of gene expression in Collaborative Cross mice}

\textbf{Chapter \ref{chap:mediation}} further develops genetic association methods in MPP, as well as the integrative mediation methodology used in \textbf{Chapter \ref{chap:hs_rats}}, though now used to test mediation at a genome-wide level. The data consist of gene expression and chromatin accessibility sequence in 47 male CC mice from lung, liver, and kidney tissues. eQTL and chromatin accessibility QTL (cQTL) were mapped using a multi-stage conditional fitting approach \citep{Jansen2017}, which allows for potentially multiple QTL to be detected per outcome given sufficient support. After QTL mapping, support for mediation of the eQTL effect on gene expression through chromatin accessibility was assessed through a genome-wide scan, similar to the approach used in \cite{Chick2016}. Given the small sample size and strong prior expectation that QTL and mediation signals will be local (arbitrarily set to within 5 Mb of outcome position), local chromosome-wide significance (based on chromosome the outcome is located on) was assessed in addition to genome-wide significance.

Though the genetic association and integrative mediation methods are largely finalized, the results are preliminary, as the investigation is ongoing and sequence data are currently being re-processed as a result of multiply-aligning sequences resulting in distal-cQTL. Large numbers of eQTL and cQTL are detected, though in reduced levels compared to humans \citep{Battle2014}, likely due to the small sample size (47 compared to 922). Within detected QTL, signals are largely local, with approximately ratios (local:distal) of 3:1 in eQTL and 4:1 in cQTL across the three tissues. This actually represents an excess in comparison to distal signal in humans (very roughly 30:1). It is likely that many distal QTL will be removed with the re-processed sequence data. Additionally, false distal signals may occur in a small sample of CC at sites with imbalanced founder haplotype contributions (described in \textbf{Chapter \ref{chap:mi}}). Classification as local is relatively strict, which strongly supports those signals as legitimate.

Strong signatures of full mediation of eQTL through chromatin accessibility are detected, which are largely local as expected. Mediation status is not equivalent to co-localization of eQTL and cQTL, which can be visualized through founder haplotype effects, further supporting the statistical mediation procedure. A similar dynamic is observed as in \cite{Battle2015} in which protein abundance QTL (pQTL) effects are less extreme than the corresponding eQTL effect, but here cQTL have larger effects than eQTL, suggesting there is some buffering of the effect of chromatin accessibility on gene expression.

\section{Final conclusion}

This dissertation represents a collection of related projects, connected by their use of MPP and focus on tailoring the statistical methodology to the unique features of such data. The underlying concepts and ideas can be re-used and extended to further expand the efficiency and efficacy of these powerful genetic resources, particularly with respect to the design of experiments, as well as genetic association and related integrative analyses.



