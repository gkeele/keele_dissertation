\begin{center}
\vspace*{52pt}
{\large \textbf{PREFACE}}
\end{center}

Here I provide additional details and brief descriptions of the chapters within the context of the overall dissertation for the published and in-preparation manuscripts that make up the chapters of this dissertation.

\textbf{Chapter \ref{chap:didact}}: This chapter began as a course project in a Bayesian statistics course, BIOS 779, in Fall 2013 at UNC, taught by Professor Amy Herring, now at Duke University. This work is an extension of Bayesian methods developed for the analysis of diallel data within the Valdar lab \citep{Lenarcic2012,Phillippi2014,Crowley2014,Maurizio2017,Turner2017}. In this work, strain-level effects are characterized from diallel data, as in \cite{Lenarcic2012}, but here these effects are used as inputs into utility functions, such that they are meaningful for QTL mapping, in order to prioritize and select bi-parental crosses. It is uniquely qualified to be the first chapter of this dissertation due to the unique intermediary position the diallel occupies between bi-parental populations and multiparental ones.

\textbf{Chapter \ref{chap:sparcc}}: This chapter began after discussions with members of the lab of Professor Samir Kelada at UNC about calculating QTL mapping power specific to the CC. Highly efficient code was developed for the QTL mapping in \textbf{Chapter \ref{chap:mediation}}, which was adapted to simulated CC data for the power calculations. Additionally, we investigated the effect of a range of genetic architectures (QTL to background strain effect sizes and allelic series) and experimental characteristics (number of CC strains and number of replicates) on power from a large scale perspective. Our goal was to provide a tool that could provide a highly tailored power for a given experiment, as well as some general power curves that can be used as reference for labs designing experiments in the CC. This work represents a bridge between the two topics of this dissertation of experimental design and genetic association in multiparental populations.

\textbf{Chapter \ref{chap:mi}}: The multiple imputation method described in detail has already been used in \cite{Mosedale2017} for QTL mapping in CC mice. This chapter is also the first chapter of this dissertation wholly focused on genetic association analyses of multiparental populations, rather than experimental design. In this work, a conservative multiple imputation approach to QTL mapping is used to avoid false associations that results from founder haplotype uncertainty and founder haplotype frequency imbalance. \textbf{Chapter \ref{chap:hs_rats}} relates to this one, as an alternative approach to QTL mapping based on the challenges that this work revealed.

\textbf{Chapter \ref{chap:hs_rats}}: In this chapter we tried multiple analytical approaches, some of which is described in \textbf{Chapter \ref{chap:mi}}, before arriving at the final process. The primary issue is that the population of heterogeneous stock (HS) rats had relatively high levels of uncertainty in terms of distinguishing founder haplotypes, as well as poor balance in terms of founder haplotype contributions. For example, at certain positions, more than half of the individuals could have inherited an allele from a single founder out of eight. We found that these joint issues led to particularly unstable haplotype-based associations (\textbf{Chapter \ref{chap:mi}}). To reduce these issues, we used an imputed SNP approach, in which we used the founder haplotype probabilities to impute SNP alleles, which effectively stabilized the associations, and even increased power by reducing the number of allele effect parameters that were being estimated. This chapter also introduces the use of mediation, which will be further used and developed in \textbf{Chapter \ref{chap:mediation}}, to better understand the biology underlying a QTL.

\textbf{Chapter \ref{chap:mediation}}: This chapter began more as a consultation on QTL mapping for collaborators in the lab of Professor Terry Furey at UNC, but became more involved as it became clear that more efficient mapping software for the CC would be required to accommodate having thousands of phenotypes (gene expression and chromatin accessibility). The work was further expanded to include assessment of the evidence for mediation of the eQTL effect on gene expression through chromatin accessibility, using a similar approach as \cite{Chick2016} used for gene expression and protein abundance. In terms of this chapter's place within the arc of this dissertation, it represents progress beyond traditional QTL mapping in multiparental populations, and additionally provides a demonstration of the value of the systems genetics approach that is possible with the CC. As the overall work is highly collaborative and unfinished, the introduction, preliminary results, and discussion will be briefer than previous chapters, and focus on the portions relevant to this dissertation.


\clearpage
