%Acknowledgements are the author's statement of gratitude to and
%recognition of the people and institutions who helped the author's
%research and writing.

\begin{center}
\vspace*{52pt}
{\large \textbf{ACKNOWLEDGEMENTS}}
\end{center}

There are many people and institutions to acknowledge and thank for supporting me in my development as a scientist capable of pursuing this research. 

First and foremost, I thank my mentor and advisor, Will Valdar. We are both self-effacing introverts, and I appreciated that we could minimally acknowledge each other in the hallway without worry that either one of us would take offense or fear that our relationship was deteriorating. In Will I found a caring advisor who fosters an environment in which his students can feel comfortable to learn and to develop, without a harsh, uncompromising focus on production. He emphasizes a comprehensive training experience, including foundations in statistics, programming, and quantitative genetics, writing, engagement within our scientific community, and how to best communicate with a given audience. I know this is not the universal graduate student experience, and am all the more grateful for it.

I am thankful for the members of the Valdar lab; undoubtedly we are an interesting cohort. Past members include Alan Lenarcic, who helped me with the analysis of diallel data, Jeremy Sabourin, whom I worked with on DiploLASSO, a cool project not included in this disseration, Zhaojun Zhang, who handed Diploffect off to me, and Yunjung Kim, who assisted in the development of the multiple imputation project. Finally Yuying Xie, who acted as my ``tiger mom" during my rotation. I miss him and his lovely family. 

As for my concurrent Valdar lab peers, they truly feel like siblings to me, meaning I care for them dearly, but they also drive me crazy. It is an injustice to our relationships to summarize things with a single quip, but alas, I must. Dan Oreper, who seems to actually appreciate my extraneous knowledge and strong feelings about animals. Robert Corty, who is the extraverted yin to my introverted yang, and can always find a pun, even if it is not really there. Paul Maurizio, who kept me safe in Italy and allowed me to be ``Tito Greg" to his son Lucas. Wes Crouse, a trophy husband who can eat a sickening quantity of hotdogs. Yanwei Cai, whom I bonded with through Pokemon, which his dog Goki basically is. Last but not least, Kathie Sun, who can really tackle an all-you-can-eat salad bar, and gives great book recommendations. I hope to keep in touch with all of them.

I would also like to thank fellow students that I interacted with while at UNC, a few whom I wish to mention further. Austin Hepperla, who survived many years of Thanksgiving turkey fry and is always willing to talk sports, politics, video games, pop culture, etc. Bryan Quach, who also survived the previously mentioned frying events, and is a wonderful collaborator on the gene expression and chromatin accessibility project, never growing angry with me as I fiddled with the QTL mapping and mediation. Natalie Stanley, who I suspect converted to Bayesian for the jokes (Seinfeld allusion). Nur Shahir, who often shares her baking with the lab, and whose laugh can be heard across the building. Lauren Donoghue, for conversations on science and shared interests (turtles, non-turtle animals, etc. and probably topics only I have interest in but she is too polite to tell me are boring). The value of friendship cannot be overstated, particularly during challenging times.

I am grateful for the education that I received at my undergraduate institution, the University of Nebraska-Lincoln, and in particular wish to thank my honors thesis advisor, John Janovy Jr. I have fond memories of listening to tales of his parasitological exploits. I am also grateful for my experiences during my Master's degree in biostatistics at the University of Michigan and in Mike Boehnke's statistical genetics group. Those years helped build the foundation in statistics and computation that have been of great value during my PhD. I also thank Jennifer Beebe-Dimmer and Ann Schwartz for the opportunity to work with them and their support at Wayne State University.

I need to thank my committee members for their interest, support, and expertise. I sincerely thank Fernando Pardo-Manuel de Villena as chair, Samir Kelada, Mike Love, Leonard McMillan, Daniel Pomp, and former members Brian Bennett and Wei Sun.

I thank my collaborators in the projects that make up this work. First I thank Leah Solberg Woods at Wake Forest University, for allowing me to analyze her heterogeneous rat data, and putting up with my mistakes and the many adventures they brought. I am grateful to Terry Furey at UNC for the opportunity to work on the gene expression and chromatin accessibility data in the CC.

I also wish to thank the Bioinformatics and Computational Biology Curriculum and the Department of Genetics, especially Cara Marlowe and John Cornett. Without them, my family may have gone uninsured, I would have unintentionally committed tax fraud, and certainly failed to file the paperwork necessary to graduate.

The studies underlying my work were made possible through funding from the Biological and Biomedical Sciences Program, the Bioinformatics and Computational Biology Training Grant, and the National Institute of General Medical Sciences.

I have also been the beneficiary of a loving family for the entirety of my life. I express love and gratitude for my siblings: Ben and his wife Christy, Marcia and her husband Nick and my niece Nadia, Alex and his wife Abby, Emma, Jeff, and Joe. Being a member of seven siblings has indelibly influenced who I have become, and though obviously I had no choice in this aspect of my identity, I cannot imagine my life without them. 

My parents have been a constant source of love and support. I am grateful for my mom, who is the nucleus of a large dispersing family. I appreciate the effort she exherts to visit regularly, and maintain a strong loving relationship with her granddaughter despite the physical distances I keep placing between them. 

I am grateful for my dad, to whom this work is dedicated and with whom it is inexorably connected in my mind. Science means many things to me. Science is a career. Science incites passion and frustration. Science has helped two quiet men, lost in their heads, talk and relate to each other. Science makes me feel like a reflection of him, reminds me that I am his son.

I express complete and utter gratitude for the love and support of my wife Lindsey. Through all my periodic anxieties and agitations, she is a steady and constant source of love and support. It seems unlikely that I will get easier to deal with, so to Lindsey, please remember me at my best and forget the exhausting moments.

Finally, I am grateful for my wonderful daughter Cori. I am proud of the work in this dissertation, but it simply cannot compare to how proud of her I am. May she always know how I adore her.

\clearpage
