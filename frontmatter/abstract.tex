%The word �Abstract� should be centered 2? below the top of the page. 
%Skip one line, then center your name followed by the title of the 
%thesis/dissertation. Use as many lines as necessary. Centered below the 
%title include the phrase, in parentheses, �(Under the direction of  
%_________)� and include the name(s) of the dissertation advisor(s).
%Skip one line and begin the content of the abstract. It should be 
%double-spaced and conform to margin guidelines. An abstract should not 
%exceed 150 words for a thesis and 350 words for a dissertation. The 
%latter is a requirement of both the Graduate School and UMI's 
%Dissertation Abstracts International.
%Because your dissertation abstract will be published, please prepare and 
%proofread it carefully. Print all symbols and foreign words clearly and 
%accurately to avoid errors or delays. Make sure that the title given at 
%the top of the abstract has the same wording as the title shown on your 
%title page. Avoid mathematical formulas, diagrams, and other 
%illustrative materials, and only offer the briefest possible description 
%of your thesis/dissertation and a concise summary of its conclusions. Do 
%not include lengthy explanations and opinions.
%The abstract should bear the lower case Roman number ii (if you did not 
%include a copyright page) or iii (if you include a copyright page).

\begin{center}
\vspace*{52pt}
{\large \textbf{ABSTRACT}}
\vspace{11pt}

\begin{singlespace}
Gregory R.\ Keele: Experimental Design \& Analysis with Multiparental Populations \\
(Under the direction of William Valdar)
\end{singlespace}
\end{center}

Multiparental populations (MPP) are experimental populations descended from more than two founder or parental inbred strains. They generally possess far greater genetic variation and phenotypic variability than simpler bi-parental populations, and are thus powerful resources for genetic studies,. MPP have been developed in numerous model systems, and have been successfully utilized in genetic association or quantitative trait locus (QTL) mapping studies for identifying candidate genes and variants that modulate complex phenotypes. Statistical methods developed for simpler populations have been extended successfully for analyses of MPP, though problems can arise, such as dubious QTL that occur at positions with imbalanced founder haplotype contributions. This shortcoming reflects the potential value of statistical tools designed specifically for MPP that can better leverage the abundant genetic and phenotypic variation for design and analyses of powerful experiments.

This dissertation has two main topics: 1) experimental design and 2) genetic association and related analyses in MPP. Within the topic of experimental design, the use of the diallel, a specific form of MPP, to inform selection of powerful follow-up bi-parental crosses for QTL mapping is explored. More broadly, this approach represents a Bayesian decision theoretic approach and is found to provide a quantitative, principled procedure for leveraging information in the pilot data towards follow-up experiments. The second design subject is a power calculation tool for the Collaborative Cross (CC), a panel of recombinant inbred strains of mice, providing highly tailored power estimates for the design of mapping studies in the realized CC strains. Additionally, the tool is used to investigate how various aspects of experimental design and features of the underlying QTL affect the power to map QTL broadly.

The first subject for the topic of genetic association is a multiple imputation approach to QTL mapping in MPP that is shown to reduce false QTL that result from founder haplotype uncertainty and imbalanced founder haplotype contributions. Next, an analysis of heterogeneous stock rats, an outbred MPP, is presented, in which imputed SNP association and fine-mapping approaches, including an integrative mediation procedure, are used to identify candidate variants influencing adiposity phenotypes. Finally, QTL mapping is performed on gene expression and chromatin accessibility outcomes in the CC, which largely detect local signals (within 5 Mb upstream or downstream of target outcome). These analyses are followed by a genome-wide integrative mediation approach, that detects local signatures of mediation of gene expression through chromatin accessibility, in a limited sample of CC mice.

\clearpage
